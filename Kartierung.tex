\subsubsection{Kartierung}
Alle Kartierungsvorgänge wurden mit der Free Open Source Software\\ \emph{QGIS}\cite{qgis}durchgeführt.
\begin{figure}[ht]
    \centering
    \includegraphics[width=.4\linewidth]{Bilder/QGIS/about-screenshot.png}
    \caption[fig:qgisabout]{QGIS-Benutzeroberfläche}
\end{figure}
\\Mithilfe dieser Software lassen sich basierend auf bereits existierenden Karten, 
wie beispielsweise \emph{OpenStreetMap}\cite{ostrm} oder \emph{OpenSeaMap}\cite{oseam}, eigene Routen und Points of \\Interest(POIs) ohne großen 
Aufwand eintragen. Genauso leicht erfolgt der Import sowie vom Multibeam,  als auch von den Navigationsgeräten der ALDEBARAN gespeicherten 
Positions- und Geschwindigkeitsdaten.\\


Basierend auf dem QGIS-Projekt von \jens, welches die \emph{OpenStreetMap},sowie den Munitionslagerstättendaten von den Webseiten
\emph{AmuCad}\cite{amucad} und \emph{Munition im Meer}\cite{muninmeer}
als Basis benutzt, konnten wir unsere Route sans Wasserprobenpunkten planen, diese wurden erst Post-Fahrt eingetragen.
% TODO : amucad und munimmeer in bib einfügen!
\begin{figure}[H]
    \begin{minipage}{0.48\textwidth}
        \centering
        \includegraphics[width=.9\linewidth]{Bilder/platzhalter.jpeg}
        \caption[fig:planned_route]{Geplante Route}
    \end{minipage}
    \begin{minipage}{0.48\textwidth}
        \centering
        \includegraphics[width=.9\linewidth]{Bilder/platzhalter.jpeg}
        \caption[fig:actual_route]{Gefahrene Route}
    \end{minipage}
\end{figure}
% Karten füge ich dann wieder ein, wenn ich meinen stick wiederfind, sorry guys