% Activate the following line by filling in the right side. If for example the name of the root file is Main.tex, write
% "...root = Main.tex" if the chapter file is in the same directory, and "...root = ../Main.tex" if the chapter is in a subdirectory.
 
%!TEX root =  TNTinderSee.tex

Da das vollständige Abfahren des Gebiets mithilfe des ROVs zeitlich und stromtechnisch nicht 
möglich gewesen wäre, brachten die Forscher des Geomar ein Multibeam mit auf die Aldebaran. \\

Ein Multibeam funktioniert ähnlich wie ein Echolot, es sendet eine Schallwelle in die zu messende 
Richtung und wartet darauf dieses Signal wieder zurückzubekommen. Durch den Zeitunterschied 
kann die Entfernung bestimmt werden. Der größte Unterschied zwischen einem Echolot und dem Multibeam 
liegt bei der Anzahl der "Beams". \\

Das von uns verwendete Multibeam konnte mit 512 Einzel-Echoloten großflächig den Meeresboden kartieren und dies 
deutlich genauer als unsere ROV-Kamera es hätte schaffen können. Da die zu messenden Schallwellen des Multibeams 
sich kegelförmig ausbreiten kann man die Genauigkeit des Multibeams in Quadratgrößen angeben. 
Wir stellen die Genauigkeit des Multibeams, später Grid gennant, zu Anfang auf 25cm$^2$ um den Computer nicht zu überlasten, aber 
dennoch noch kleine Gegenstände zu finden. Wichtig ist hierbei, wir verstellen nur die angeziegten Daten, die Rohdaten werden aufgezeichnet. 
So können wir später die Auflösung an das echte Maximum anpassen.\\

Da die Schallgeschwindigkeit in jedem Gewässer leicht unterschiedlich ist, wurde auf dem Multibeam
ein Schallgeschwindigkeitsmesser montiert. Doch nicht nur hier kann sich die Geschwindigkeit ändern!
Auch innerhalb des Wassers gibt es Schichten, in denen der Schall sich langsamer fortbewegt als in
der Schicht darüber oder darunter. Dadurch kommt es zu Brechungen des "Beams" und der tatsächliche 
Strahlungswinkel des Multibeams verändert sich. Dies hat dann wiederrum einen Effekt auf die finale Kartierung.
Ab einer Tiefe von mehr als 10 Meter muss dieser Effekt beachtet werden. Da unser Forschungsgebiet nur eine 
Tiefe von maximal 7 Meter hat, können wir diesen Effekt vernachlässigen.\\

Jens Greinert und Mareike Kampmeier, beide Mitarbeiter des Geomar, stellen uns das Multibeam zur Verfügung und 
leisten eine großartige Hilfe bei der Kalibrierung und Datenverwertung des selbigen. Die gewonnen Daten werden 
später auch vom Geomar verwendet und analysiert werden. \\

Die Kalibrierung umfasst den Roll, Pitch und Yaw des Schiffes mithilfe eines Bewegungssensors aus den Daten zu 
entfernen und auch die Befestigung des Multibeams am Schiff zu sichern. Da diese Befestigung mobil sein muss, kann
dies nicht so genau wie an einer festinstallierten Stelle passieren. Die Ergebinsse sond dennoch hoch genau. 
Das Multibeam ist mithilfe eines Metallgestells an der Linken Seite der Aldebaran angebracht und kann bei Bedarf 
abgesenkt oder hochgezogen werden.\\

Wir fahren möglichst parallele, gerade Bahnen mit dem Multibeam um eine möglichst hohe Abdeckung zu erreichen. 
Das wir, um die Genauigkeit des Multibeams nicht zu gefährden, nur knapp 3 Knoten schnell fahren dürfen setzt
uns jedoch dem starken Wind aus. \\

Auch die Befestigung des Multibeams bachtet uns hin und wieder leichte Fehler in die Kartierung. Für die 
Umstände war das Multibeam trotzdem erstaunlich genau und wir konnten drei interessante Punkte auf dem 
Meeresboden ausfindig machen. \\

Im Geomar können wir die Rohdaten analysieren. Auch das "Grid", können wir nun verfeinern von 25cm$^2$ auf 10cm$^2$. 
Wir löschen fehlerhafte Daten, erstellen eine 3D-Karte des gescannten Gebiets und erzeugen ein "Backscatter". 
Das "Backscatter" umfasst weniger die Tiefe des Gebiets als die empfangene Laustärke unserer Signale. 
Diese Lautstärke wird von der Tiefe des Gebiets und vor allem von der beschaffenheit des Bodens bestimmt. \\

Generell kann man sagen, das je härter der Boden ist, desto stärker werden die Schallwellen reflektiert. 
Auf der "Backscatter"-Karte wird diese Laustärke veranschaulicht und man kann die Art des Bodens bestimmen.
Auch Munition würde auf dieser Karte stark auffallen, da Sie die Schallwellen starkreflektiert und so 
auf der Karte hell aufleuchten würde. Auf der von uns erzeugten Karte erkennt man die Seegraßwiesen in 
der Nähe der Insel Vilm an ihrer geringen Reflektion.\\

Große Bruchstücke der Schuten oder herumliegende Munition konnten wir auf dem Multibeam keine erkennen und 
auch mit dem Tauchroboter stellten sich die interessanten Punkte als ungefährlich heraus. \\
