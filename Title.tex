% Activate the following line by filling in the right side. If for example the name of the root file is Main.tex, write
% "...root = Main.tex" if the chapter file is in the same directory, and "...root = ../Main.tex" if the chapter is in a subdirectory.
 
%!TEX root =  TNTinderSee.tex

% mehrere Bilder in einer Bildumgebung mit 
%\subfloat{bild.jpg}


\begin{titlepage}
\thispagestyle{empty}
 \begin{center}
 \begin{figure}[htbp]
    \centering
 %   \subfloat{\includegraphics[width=0.15\textwidth]{Bilder/Titel/bild-firma.JPG}}\quad
   %  \subfloat{\includegraphics[width=0.25\textwidth]{Bilder/Titel/bild-uni 
%oder fh.JPG}}
\end{figure}
\vspace*{1cm}
 \Large{Schiller-Gymnasium Offenburg }

  \vspace*{1.5cm}
 {\huge Thema}
 \vspace*{1cm} \\
 {\Large Abschlussbereicht\\von\\Eitel, Martin \\ \texttt{marjelly1@gmail.com}\\Komyakov, Alexander \\\texttt{alexander.komyakov@lynxisgod.eu} \\ Rehwinkel, Antonio \\ \texttt{antonio.rehwinkel@schiller-og.de} \\ Sauerbrey, Luisa \\ \texttt{luisa.sauerbrey@schiller-og.de}
 \vspace{0.5cm}
 {\Large \bfseries \\}
 \vspace{0.5cm}
 {\Large geboren am dein Geburtstag}
 \vfill
  \vspace*{1.5cm}
\begin{table}[h]
	\centering
	\begin{tabular}{|l| l|}\hline
		Aufgabensteller & Prof. \\ \hline
		Durchgeführt bei: & Firmenname\\ \hline
		Betreuer: & Betreuer Firma\\ Marek Czernohous \\ \hline
		Wissenschaftspate & \\ Prof. Dr. Jens Greinert\\ \hline
		Arbeit vorgelegt am: & Datum\\ \hline
	\end{tabular}
\end{table}
}
\end{center}
\end{titlepage}

