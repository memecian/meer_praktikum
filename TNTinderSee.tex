\documentclass[12pt,titlepage]{scrbook}
\usepackage[ngerman]{babel}
\usepackage[utf8]{inputenc}
\usepackage{color}
\usepackage[a4paper,lmargin={4cm},rmargin={2cm},tmargin={2.5cm},bmargin = {2.5cm}]{geometry}
\usepackage{amssymb}
\usepackage{amsthm}
\usepackage{graphicx}


\begin{document}

\begin{titlepage}
\title{Die Auswirkung von Munitionshalden auf die Wasserqualität der Ostsee zwischen Vilm und Lauterbach}
\date{01.08.2021}
\author{Martin Eitel, Alexander Komyakov, Antonio Rehwinkel, Luisa Sauerbrey}
\maketitle
\end{titlepage}
\tableofcontents
\chapter{Vorwort}
\section{Forschungsgegenstand}
\section{Erwartungen}
\chapter{Hauptteil}
\section{Kartierung}
    \subsection*{QGIS}
    Hier kommt das eine Foto von mir mit dem Laptop, und ein text zu QGIS.
\section{Wasserproben}
\section{Sedimentproben}
\section{Multiparametermessungen}
\section{ROV-Kartierungen}
\section{Fazit}

\end{document}