
Unsere Grundlegende Idee zur ROV Analyse war vollgende, erst wollten wir das als Munitionsbelastete gekennzeichnete Gebiet mithilfe 
des Multibeams grob Katieren und nach genauere Betrachtung 
Interessante bzw. auffällige Stellen finden. Daraufhin wollten wir diese mit unserem ROV antauchen um diese genauer Bestimmen zu können.
Dieses Vorgehen konnten wir auch so zum großteil durchführen.
Die genauer zu bestimmenden Stellen haben wir mit unserem eigenen ROV (Remotely Operated Vehicle) eine Art Tauchroboter angetaucht. 
Dieser entstand vor 3 Jahren in unsere Forschungsgruppe aus einem Bausatz der Firma BlueRobotics und trägt die Bezeichnung "Blue ROV 2".
Der Tauchrobter lässt sich mithilfe eines Laptops und eines Controllers über ein langes Kabel (auch Tether genannt) frei in alle Richtugen in der Wassersäule bewegen.
Und zusammen mit der schwenkbare live Kamera war unser ROV das Perfekte Inspektionsfahrzeug für unsere Forschungziel. Der ROV wird optimaler weise 
von vier Personen bedient da zwei Personen die Tetherspule bedienen müssen, man einen Navigator benötigt, da die Orientierung Unterwasser in einem fremdem Gebiet sehr anspruchsvoll Inspektionsfahrzeug und das ROV von einem Piloten Gesteuert werden muss.
Somit beansprucht das ROV fahren unser gesamtes Forschungsteam und die Restlich Crew um das Boot auf position zu Halten, wodurch das ROV fahren nicht nur einen hohen zeitlichen aufwand hat, sondern auch die gesamte Kapazität der Crew benötigt.
Aus diesem Grund war es noch viel wichtiger geziehlte Stellen ausfindig zu machen an denen wir mit dem ROV abtauchen wollten und nicht nur blind im Meeresboden rumstochern.
\\

Das ROV von BLueRobotics besteht aus zwei runden Druckgehäusen eine für den Lithium-Ionen Akku, die andere für die Steuerungselektonik und die Kamera.
Vor jedem Tauchgang müssen diese Druckgehäuse mithilfe einer Vakumpumpe auf ihre dichtigkeit überprüft werden. 
Daraufhin kann das ROV mithilfe des Tethers an den Laptop angeschlossen werden, auf dem sich die Steuerungssoftware befindet.
Um die Abläufe des ROV tauchens einzuspielen und den ROV auf der Aldebaran zu testen haben wir am Sonntag Abend einen Testtauchgung im Hafen durchgeführt. 
Hierbei ist uns schon die Schlechte Sicht in der Ostee aufgefallen, die uns später noch einige Schwierigkeiten machen sollte.
\\

Insgesamt haben wir für unsere Forschung auf der Aldebaran an drei Unterschiedlichen Stellen Insgesamt Fünf ROV Tauchgänge durchgeführt 
%Einfügen Karte mit Tauchstellen des ROV mit Objekten auf dem Multibeam
Während diesen Tauchgängen und bei deren Vorbereitung hatten wir mehrere Schwierigkeiten die aufgrund unserer Begrenzten Zeit und Kapazität unsere Forschung verkompliziert hat.
Da die Aldebaran ein verhältnissmäßig kleines Schiff für ein Forschungsschiff ist, hat sie zwar den Vorteil dass sie gerade durch das einklappbares SChwert einen sehr geringen Tiefgang von 80 cm hat, was uns die Forschung in diesem Gebiet erst ermöglicht hat.
Anderrerseits ist sie deshalb auch sehr Anfällig gegenüber Wind und Wetter, weshalb das Arbeiten an Deck bei den Zeitweise aufgetretenen Windstärken von vier bis fünf erschwert wurde.
Dies hat sich sowohl bei der Vorbereitung des ROVs gezeigt als auch bei den Tauchgängen, da die Aldebaran trotz Anker hin und her pendelte, wodurch sich die Position ständig verändert hat.
Die Navigation und die Orientierung Unterwasser mit dem ROV war mit Abstand die größte herausforderung, da die einzige möglichkeit sich zu Orientieren der eingebauter Kompass ist.
Und der ROV aufgrund seiner Größe kein USBL (Ultra Short Baseline) - System besitzt, wodurch die genaue Position des ROVs nicht bekannnt ist.
Deshalb ist die einzige möglichkeit eine genaue position, dort wo das auffälige Objekt durch das Multibeam ausgemacht wurde, anzutauchen, indem man durch die Bekannte Position des Schiffes und somit den Startpunktes des ROVS, das Objekt anpeilt und den ROV genau auf diesem Kompasskurs hält.
Was hierbei nicht ganz trevial war dass das Schiff durch das Pendeln stänfig seine Position veränderte.
Den ROV auf einem genauen Kurs zu halten war hierbei problematisch, da durch die geringe Tauchtiefe von ca. 5m das ROV von den Wellen und dem Sturm hin und her geworfen wurde und abgedriftet ist.
Was zusätzlich noch schwierigkeiten bereitet hat war die empfindliche Steuerung, mit der es kompliziert war den ROV auf einer geraden Linie zu halten.
Die interessanten Objekte haben wir mithilfe der optischen Kamera untersucht, jedoch durch die sehr geringe Sichweite in der Ostesee wurde dies sehr erschwert, da man an den Grund bzw. an das Objekt bis auf eine nähe von ca. 20 cm herantauchen musste um das Objekt zu Identifizieren. 