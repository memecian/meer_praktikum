% Activate the following line by filling in the right side. If for example the name of the root file is Main.tex, write
% "...root = Main.tex" if the chapter file is in the same directory, and "...root = ../Main.tex" if the chapter is in a subdirectory.
 
%!TEX root =  TNTinderSee.tex

\chapter[Einleitung]{Einleitung}

Im Meer lagernde Munition stellt eine Gefahr dar. Nicht unbedingt durch unmittelbare Detonotationsgefahr, 
sondern durch die langsame Zersetzung, die die enthaltenen Sprengstoffe nach und nach 
freilegt\cite{zeitbomben}. Die Entstehung sprengstofftypischer Abbauprodukte, sowie das direkte Austreten 
giftiger Stoffe, stellen Gesundheitsgefährung exponierter Meerestiere, aber auch Menschen dar, denn die
Abbauprodukte gelten als krebserregend und das potentiell austretende Phosphor lagert sich an den Stränden 
ab und ist von Bernstein kaum zu unterscheiden. Als wir den Meereswettbewerb mit der Aldebaran fanden, 
fühlten wir uns gezwungen nachzuforschen wie groß die Gefahr schon heutzutage ist.\\

Wir wollten in einem potentiellem Munitionsabwurfsgebiet messen, wie groß die Anteile der Schadstoffe sind und 
ob noch etwas von den Granaten und Bomben zu sehen ist. Dazu entschieden wir uns für ein Gebiet kurz Vilm,
da hier angeblich zwei Schuten nach dem Zweiten Weltkrieg mit Munition beladen explodiert sein sollen.\\

Das Naturschutzgebiet Vilm liegt in etwa 100 Meter Entfernung zu der Explosionsstelle und auch weiter 
entfernte Orte können durch Ablagerungen und Schadstoffe in Fisch-Fängen betroffen sein.\\

In Zusammenarbeit mit der Aldebaran und dem Geomar wollten wir die Wracks kartieren und die Schadstoffwerte in
der direkten Umgebung messen. Besonders interessant wäre die beförderte Munition gewesen. Die Frage der
Ortung und Untersuchung der möglichen Funde konnten wir zusammen mit dem Geomar lösen. Mithilfe eines 
"Multibeams" sollten wir interessante Orte ohne Tauchgang finden können um später mit dem Schuleigenen 
ROV die Orte genauer unter die Lupe nehmen zu können. Auch Sedimentproben dirket neben den Fundstätten 
konnten wir mithilfe eines eignene Aufsatzes realisieren.

Wir befürchteten hohe Werte für Phosphor und sprengstofftypische Verbindungen wie Trinitrotoluol (TNT) 
zu messen und unter Umständen sogar Granaten oder Bomben zu finden. Falls dies passieren sollte, würde 
sich für uns die Frage stellen, wie sich diese Munition auf die Öknomie ausgewirkt hat. 

Themenfindung, Relevanz, entscheidende Fragen, Lösungsansätze, Forschungsstand mit Quellen, Erwartungen, konkrete Fragen.
