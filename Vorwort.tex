% Activate the following line by filling in the right side. If for example the name of the root file is Main.tex, write
% "...root = Main.tex" if the chapter file is in the same directory, and "...root = ../Main.tex" if the chapter is in a subdirectory.
 
%!TEX root =  TNTinderSee.tex

\chapter[Einleitung]{Einleitung}

Im Meer lagernde Munition stellt eine Gefahr dar. Nicht unbedingt durch unmittelbare Detonationsgefahr, 
sondern durch die langsame Zersetzung, die die enthaltenen Sprengstoffe nach und nach 
freilegt\cite{zeitbomben}. Die Entstehung sprengstofftypischer Abbauprodukte, sowie das direkte Austreten 
giftiger Stoffe, stellen Gesundheitsgefährung exponierter Meerestiere, aber auch der Menschen dar\cite{spread}, denn die
Abbauprodukte gelten als krebserregend und das potentiell austretende Phosphor lagert sich an den Stränden 
ab und ist von Bernstein kaum zu unterscheiden.  Im Rahmen des Meereswettbewerbes der Deutschen Meeresstiftung ergab sich für uns die einmalige Möglichkeit, konkrete Messungen vor der Küste Rügens, durchzuführen, um zu prüfen, wie hoch die konkrete Gefahr in der mutmaßlichen Nähe von Sprengstoffresten ist.\\

Konkret wollten wir erforschen, wie konzentriert etwaige Schadstoffe im Wasser zu finden sind und 
ob noch etwas von Granaten und Bomben zu sehen ist. Wir entschieden uns für ein Gebiet kurz vor Vilm,
da hier angeblich zwei Schuten nach dem Zweiten Weltkrieg mit Munition beladen explodiert sein sollen.\cite{schiffsschicksale}\\

Das Naturschutzgebiet Vilm liegt in etwa 100 Meter Entfernung zu der vermuteten Explosionsstelle, wobei auch 
entfernte Orte durch Ablagerungen und Schadstoffe in Fisch-Fängen betroffen sein können.\\

In Zusammenarbeit mit der Forschungsyacht Aldebaran und dem GEOMAR Hemlholtzzentrum für Ozeanforschung Kiel, wollten wir die Wracks kartieren und die Schadstoffwerte in
der direkten Umgebung messen. 
Mithilfe eines 
\glqq Multibeams\grqq, welches das GEOMAR zur Verfügung gestellt und mit uns betrieben hat, sollten wir interessante Orte ohne Tauchgang finden können, um später mit dem schuleigenen Unterwasserroboter (Remote Operated Vehicle - ROV)  und Kamera die Orte genauer unter die Lupe nehmen zu können. Auch Sedimentproben und Wasserproben konnten wir direkt neben den Fundstätten entnehmen und mit Hilfe einer Ionenchromatografie analysieren.\\

Wir befürchteten hohe Werte für Phosphor und sprengstofftypischen Verbindungen wie Trinitrotoluol (TNT) 
zu messen und unter Umständen sogar Granaten oder Bomben zu finden. Falls dies passieren sollte, würde 
sich für uns die Frage stellen, wie sich diese Munition auf die Ökoomie ausgewirkt hat. \\

In schon existierenden Forschungen zeigt sich, die abgesonderten Schadstoffe verursachen gesundheitliche Schäden bei Fischen 
und somit in unserer Nahrungskette.\cite{munitionsbelastung} 

Aber, wie stark ist die Belastung vor der Südostküste Rügens?
%https://www.schleswig-holstein.de/DE/UXO/EN/Themes/Subjects/assessements.html
%
%https://www.geomar.de/fileadmin/content/service/presse/Pressemitteilungen/2020/pm_2020_37_munition-sh.pdf
%
%https://www.eskp.de/schadstoffe/wie-weiter-mit-den-kampfmittelaltlasten-im-meer-9351103/
%
%https://www.eskp.de/schadstoffe/wie-weiter-mit-den-kampfmittelaltlasten-im-meer-9351103/

%Themenfindung, Relevanz, entscheidende Fragen, Lösungsansätze, Forschungsstand mit Quellen, Erwartungen, konkrete Fragen.


% https://www.geomar.de/fileadmin/content/service/presse/Pressemitteilungen/2020/pm_2020_37_munition-sh.pdf



