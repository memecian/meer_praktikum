% Activate the following line by filling in the right side. If for example the name of the root file is Main.tex, write
% "...root = Main.tex" if the chapter file is in the same directory, and "...root = ../Main.tex" if the chapter is in a subdirectory.
 
%!TEX root =  TNTinderSee.tex

\chapter[Danksagungen]{Danksagungen}
Wir danken Prof Dr. Jens Greinert vom GEOMAR Kiel und Leiter der Arbeitsgruppe DeepSea-Monitoring. Er ist einfach der beste Wissenschaftspate, den wir uns hätten wünschen können.\\


Frank Schweikert und Dr. Hannes Imhof von der deutschen Meeresstiftung, für die tollen Tage und all die Hilfe an Board der Aldebaran. \\


Mareike Kampmeier vom GEOMAR Kiel für die geduldige Einführung am Multibeam und die vielen Erklärungen zum Thema Sprengstoffe im Wasser. \\


Dr. Inken Suck vom GEOMAR Kiel, der bestimmt coolsten ROV-Piloten dafür, dass sie uns gezeigt hat, wie man Unterwasserfahrzeuge richtig navigiert. \\


Maria Martinez Cabanas vom GEOMAR Kiel dafür, dass sie uns und unsere Proben in Ihre Labore mitgenommen hat, und geduldig stundenlang alle Fragen beantwortet hat.\\

Dr. Kevin Köser vom GEOMAR KIel für den inspirierenden Vortrag über Photogrammetrie \\


Yifan Song vom GEOMAR Kiel dafür, dass er uns ganz praktisch beigebracht hat, wie man Videomaterial georeferenziert.



Frank Schweikert und Dr. Hannes Imhof von der deutschen Meeresstiftung, für die tollen Tage und all die Hilfe an Board der Aldebaran. \\



Mareike Kampmeier vom GEOMAR Kiel für die geduldige Einführung am Multibeam und die vielen Erklärungen zum Thema Sprengstoffe im Wasser. \\



Dr. rer. nat. Inken Suck vom GEOMAR Kiel, der bestimmt coolsten ROV-Piloten dafür, dass sie uns gezeigt hat, wie man Unterwasserfahrzeuge richtig navigiert. \\



Maria Martinez Cabanas vom GEOMAR Kiel dafür, dass sie uns und unsere Proben in Ihre Labore mitgenommen hat, und geduldig stundenlang alle Fragen beantwortet hat.\\



Yifan Song vom GEOMAR Kiel dafür, dass er uns ganz praktisch beigebracht hat, wie man Videomaterial georeferenziert.\\



Besonders beeindruckt hat uns, dass uns die Wissenschaftler:innen wie selbstverständlich Samstag und Sonntag zur Seite standen und uns alles gezeigt haben.\\



Vielen Dank auch an Svenja Ehlers vom GEOMAR für ihre Hilfe und Geduld was das administrative der Reise anging und natürlich auch an Sofie Steinhausen von der deutschen Meeresstiftung, für die stets hilfsbereite Begleitung während des ganzen Wettbewerbes.\\



Dr. Kevin Köser vom GEOMAR Kiel für den inspirierenden Vortrag über Photogrammetrie und Lichtverhältnisse unter Wasser.\\



Yifan Song vom GEOMAR Kiel dafür, dass er uns ganz praktisch beigebracht hat, wie man Videomaterial georeferenziert.\\



Marek Czernohous für die insgesamt über 24 Stunden Fahrzeit, die Mate, Franzbrötchen und enorme Hilfe beim \LaTeX -Satz.\\


